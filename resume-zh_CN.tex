% !TEX TS-program = xelatex
% !TEX encoding = UTF-8 Unicode
% !Mode:: "TeX:UTF-8"

\documentclass{resume}
\usepackage{zh_CN-Adobefonts_external} % Simplified Chinese Support using external fonts (./fonts/zh_CN-Adobe/)
% \usepackage{NotoSansSC_external}
% \usepackage{NotoSerifCJKsc_external}
% \usepackage{zh_CN-Adobefonts_internal} % Simplified Chinese Support using system fonts
\usepackage{linespacing_fix} % disable extra space before next section
\usepackage{cite}

\begin{document}
\pagenumbering{gobble} % suppress displaying page number

\name{林泽佳}

\basicInfo{
  \email{linzj39@mail2.sysu.edu.cn} \textperiodcentered\ 
  \github[zeege-0.github.io]{https://zeege-0.github.io}

  % \phone{(+86) 131-221-87xxx} \textperiodcentered\ 
  % \linkedin[billryan8]{https://www.linkedin.com/in/billryan8}
}
 
\section{\faGraduationCap\  教育背景}
\datedsubsection{\textbf{中山大学}, 广州}{2022 -- 至今}
\textit{学硕}\ 计算机科学与技术 \qquad 导师:张献伟
\datedsubsection{\textbf{西北工业大学}, 西安}{2018 -- 2022}
\textit{本科}\ 软件工程 \qquad\qquad\qquad 排名:22 / 248 \qquad\quad 英语六级:610 / 750
\datedsubsection{\textbf{广州市执信中学}, 广州}{2015 -- 2018}
\textit{高中}\ 

\section{\faUsers\ 项目经历}

\datedsubsection{\textbf{基于编译器的核函数多流调度, 中山大学}}{2022 -- 至今}
ICCD'23在投论文,使用编译器分析核函数数据依赖,自动进行多流调度。\\与GrCUDA和Taskflow相比取得平均1.31x和1.16x加速比,只需额外引入2.3\%行代码。

\vspace*{0.3em}
\datedsubsection{\textbf{编译原理课程助教, 中山大学}}{2023.03 -- 2023.07}
实现基于LLVM的C语言子集的编译器,负责抽象语法树$\to$ IR生成部分。

\vspace*{0.3em}
\datedsubsection{\textbf{GPU加速的视频盲水印, 中山大学}}{2022.12}
课程设计,使用CUDA实现混合精度和多流并行的盲水印嵌入/提取算法。\\在A100上对2K视频嵌入/提取速度达到8.1/17.9FPS。

\vspace*{0.3em}
\datedsubsection{\textbf{微信支付后台开发实习, 腾讯(深圳)}}{2021.07 -- 2021.09}
使用Django搭建分布式的企业微信机器人,自动回复后台测试环境相关问题。

\vspace*{0.3em}
\datedsubsection{\textbf{复合材料构件的智能无损检测, 西北工业大学}}{2020.06 -- 2022.05}
项目负责人,入选省级大创项目,设计缺陷图像分割模型。\\使用仅1.4M参数取得60.82\%的Dice系数,在MI100上对1080p图像分割推理速度达到237FPS。


\section{\faBook\ 发表论文}

\begin{itemize}
  \item $[\rm{CF'22}]$ \href{https://dl.acm.org/doi/10.1145/3528416.3530231}{moTuner: A Compiler-based Auto-tuning Approach for Mixed-precision Operators} \\ Zewei Mo, \underline{\textbf{Zejia Lin}}, Xianwei Zhang, and Yutong Lu
\end{itemize}

% Reference Test
%\datedsubsection{\textbf{Paper Title\cite{zaharia2012resilient}}}{May. 2015}
%An xxx optimized for xxx\cite{verma2015large}
%\begin{itemize}
%  \item main contribution
%\end{itemize}

% \section{\faCogs\ 技能}
% \begin{itemize}[parsep=0.5ex]
%   \item 编程语言: C/C++, CUDA, Python, Java
%   \item 英语六级:610/750
% \end{itemize}

\section{\faHeartO\ 获奖情况}

\begin{onehalfspacing}
\datedline{\textit{中山大学}\ 一等奖学金 (夏令营排名第5)}{2022学年}
\datedline{\textit{西北工业大学}\ 优秀本科毕业设计}{2022年}
\datedline{\textit{西北工业大学}\ 校级一等奖学金}{2018, 2019, 2021学年}
\datedline{\textit{中国高校计算机大赛微信小程序应用开发赛}\ 全国二等奖 (5\%, 参赛队长)}{2020年8月}
\end{onehalfspacing}

% \section{\faInfo\ 其他}
% % increase linespacing [parsep=0.5ex]
% \begin{itemize}[parsep=0.5ex]
%   \item 技术博客: http://blog.yours.me
%   \item GitHub: https://github.com/username
%   \item 语言: 英语 - 熟练(TOEFL xxx)
% \end{itemize}

%% Reference
%\newpage
%\bibliographystyle{IEEETran}
%\bibliography{mycite}
\end{document}
